\documentclass{article}

\begin{document}

\title{Isolation of a {\em Shewanella} species that grows
  anaerobically on nitrate and acetate and aerobically on LB}
\author{C. Titus Brown}
\date{August 15, 2016}

\maketitle

\section*{Introduction}

\section*{Methods}

\paragraph{16s Colony PCR} Each colony was picked and touched to 20ul of
alkaline PEG200 (``ALP'').  This was then boiled for 5 minutes to lyse
the cells. 1 ul of cell lysate was then added to a 25 ul Promega GoTaq
G2-based PCR reaction with the bacterial 16s primers 8F and 1391R.  16
rounds of PCR was performed with a 1.5 minute extension time.  After
PCR, bands were checked on a gel; for successful amplifications,
Sanger sequencing was performed directly from the PCR reactions with
the primer 515F.

\section*{Results and Discussion}

\subsection*{An enrichment from Trunk River grew with the addition of acetate}

I initially designed media to enrich for sulfur oxidizing denitrifiers
microbes (``BB+SN'').  I inoculated two Pfennig bottles (approx. 40ml)
with approximately 2-3 cc of material.  The material for the first
enrichment, culture 4, was taken directly from the sediment layer on
top of the sand in the main channel of Trunk River, approximately 8m
from the start of the narrow outflow channel. The material for the
second enrichment, culture 8, was taken from the underwater surface of
a sea table enrichment that originated from a microbial mat, also
taken from Trunk River.  Both enrichments were incubated at 30 deg
in a foil-lined box.

After 18 hours, significant turbidity was observed in both
enrichments, along with substantial amounts of supersaturated gas,
indicating growth.  I therefore transferred 1 ml from each enrichment
to another Pfennig bottle containing BB+SN.  These transfer
enrichments, however, failed to grow.

Based on scent, Dr. Leadbetter suspected that the original transfer of
sediment contained acetic acid, indicating the presence of significant
amounts of acetate.  I therefore added 800ul of 1M sodium acetate to
both enrichment cultures, to a final concentration of 20 mM.

After the addition of acetate, both transfers grew to white opacity
within about 16 hours at 30 deg.  Two subsequent transfers of each
culture (1ml into 40 ml ``BB+SNA'', BB with thiosulfate, nitrate, and
acetate) also exhibited the same growth.

\subsection*{Enrichments exhibited nitrate and acetate dependent growth}

To further analyze growth conditions, I employed a simple
``differential diagnosis'' approach and transferred each enrichment to
four culture conditions: BB, BB with thiosulfate and acetate (BB+SA),
BB with nitrate and acetate (BB+NA), and BB with acetate (BB+A).
After incubation for 16 hours, only the BB+NA grew, indicating that
the enrichments required both nitrate and acetate but not
thiosulfate. @@

\subsection*{Enrichments yielded colonies when grown on solid media}

I plated 1:500 and 1:5000 dilutions of enrichments 4 and 8 on BB+SNA
solid media, and incubated the plates both anaerobically and
aerobically at 30 deg.  I also plated the same dilutions on LB and
grew aerobically at 30 deg.  All plates showed density-dependent growth,
although the colonies on the LB plates grew much faster (2-4 times)
than either the aerobic or anaerobic BB plates.

\subsection*{Isolate colonies from aerobic LB plates grew successfully
  in anaerobic culture}

I picked 4 colonies grown aerobically on LB from each enrichment (for
a total of 8), and inoculated anaerobic BB+SNA cultures with them.
7/8 of the cultures grew within 48 hours, with three (culture 3 from
enrichment 8, and cultures 5 and 8 from enrichment 4) growing
overnight to opacity.

I then transferred these three isolates (3, 5, and 8) from BB+SNA
anaerobic liquid culture back to LB plates, where they again grew
(@@).

\subsection*{Isolate colonies were identified by 16s colony PCR as {\em Shewnaella} spp.}

I extracted DNA from the eight LB isolate colonies by picking the
colonies and boiling them in 20ul of ALP, and then used
the @@ and @@ 16s primers to amplify a diagnostic region of the 16s
gene with PCR.  All 8 yielded bands, which were then Sanger sequenced.
BLAST of all 8 colonies against nt revealed @@@ (Table @@@).  Colonies
3, 5 and 8 showed strong similarity to XX YY ZZ, clearly identifying
the colonies as members of the genus {\em Shewanella}.
%(decolorationis and algae).

\begin{table}
\centering
\begin{tabular}{|c|c|c|l|l|}
\hline
Enrichment & Colony & Match percentage & BLAST match & Accession \\
\hline
1 &
8 &
99\% &
Shewanella algae strain MAS2736 &
GQ372874.1 \\

2 &
8 &
99\% &
Shewanella algae strain MAS2736 &
GQ372874.1 \\

3 &
8 &
99\% &
Shewanella algae strain MAS2736 &
GQ372874.1 \\

4 &
8 &
99\% &
Shewanella algae strain MAS2736 &
GQ372874.1 \\

5 &
4 &
99\% &
Shewanella sp. Chr-15 &
JQ863373.1 \\

6 &
4 &
99\% &
Shewanella sp. Chr-15 &
JQ863373.1 \\

7 &
4 &
99\% &
Shewanella sp. Chr-15 &
JQ863373.1 \\

8 &
4 &
99\% &
Shewanella sp. Chr-15 &
JQ863373.1 \\
\hline
\end{tabular}
\caption{BLAST-based characterization of Sanger-sequenced 16s regions from colony PCRs (position 515 and 3' sequence).  All colonies are clearly identified as Shewanella at 99\% identity.}
\label{tab:16s}
\end{table}

\subsection*{Isolate cultures contain Spirilla-shaped microbes}

I examined all three cultures with light microscopy and saw
Spirilla-like bacteria (fig @@).  These bacteria appeared to move in a
corkscrew-like fashion.  This morphology is at odds with their
molecular identification as Shewanella spp, which are typically
rod-shaped bacilli.

\end{document}
